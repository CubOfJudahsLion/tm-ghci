\documentclass{article}
\usepackage{parskip,graphicx,hyperref,xcolor,ulem}

\newcommand{\TeXmacs}{\includegraphics[width=1.1cm,height=0.25cm]{texmacs-text.png}}
\renewcommand{\emph}[1]{\textit{#1}}

\begin{document}

\title{\texttt{tm-ghci\ $::$\ }
  \raisebox{0.0\height}{\includegraphics[width=0.5cm,height=0.5cm]{haskell.png}}
  $\mathtt{\rightarrow}$\ 
  \raisebox{0.0\height}{\includegraphics[width=0.5cm,height=0.5cm]{texmacs.png}}}

\author{}
\date{}

\maketitle

A \href{https://texmacs.org/}{\TeXmacs} plugin for running
\href{https://wiki.haskell.org/GHC/GHCi}{\textsc{GHCi}} sessions.

\section{Installing}

\subsection{From release files}

The release archives contain statically linked files, which makes it possible
to interact with different versions of {\textsc{GHCi}}.

To install, uncompress the archive for your \textsc{OS} at
\texttt{\$TEXMACS\_PATH/plugins/}  (usually
\texttt{/usr/share/TeXmacs/plugins/}  in  \textsc{Linux},
\emph{\textless\textsf{TeXmacs-install-dir}\textgreater}\texttt{\textbackslash
plugins\textbackslash} in \textsc{Windows}) or
\texttt{\$TEXMACS\_HOME\_PATH/plugins/}
(\texttt{\$HOME/.TeXmacs/plugins/} in \textsc{Linux} or
\texttt{\%UserProfile\%\textbackslash AppData\textbackslash
Roaming\textbackslash TeXmacs\textbackslash plugins\textbackslash} in
\textsc{Windows}.)

\texttt{ghci} needs to be in the system \texttt{PATH} for the plugin
to work.

The plugin is rather simple --- mostly forwarding input and output. The
resulting interaction and evaluation facilities are all {\TeXmacs}'s.

\subsection{Building and installing from source}

Building the plugin requires a \texttt{bash}-style shell and a
{\textsc{Haskell}} distribution ({\textsc{GHC}} 9.2.5 or later.) In
{\textsc{Windows}}, this means installing {\textsc{Haskell}} through
{\textsc{\href{https://www.haskell.org/ghcup/}{GHCup}}} and setting up an
{\textsc{Msys2}} directory, then installing the development packages so that
\texttt{make}, \texttt{strip} and other required utilities will be
available.

Installing is simple. In your shell prompt, just \texttt{cd} to
the \texttt{ghci} folder and type:

\texttt{make deploy}

Which takes care of compiling and installing the plug-in in the appropriate
directory (\texttt{\$TEXMACS\_HOME\_PATH/plugins}.)

While this should be sufficient, this software is still in alpha phase, and
it's only been tested in {\textsc{Windows}} 10 and {\textsc{Arch Linux}}. I
don't own a {\textsc{MacOS}} system, so I'm unable to provide the respective
package.

\section{Features and Limitations}

\texttt{tm-ghci} is {\emph{alpha}}-stage software, distributed
{\emph{as-is}}.

Presently, it's able to run a vanilla \texttt{ghci} session, i.e., without
project dependencies autoloaded as with \texttt{stack repl} or
\texttt{cabal repl}.

\section{Licensing}

\texttt{tm-ghci} is distributed under the
\href{https://www.gnu.org/licenses/gpl-3.0.en.html}{\textsc{GPL3}}
license.

\section{Issues}

Please \href{https://github.com/CubOfJudahsLion/tm-ghci/issues}{submit an
issue} if you find one. Bug reports must include the steps required to
reproduce the error.

You can also send any feedback to
\emph{10951848+\sout{nope}CubOfJudahsLion} {\"a}{$\tau$}
\emph{users.noreply.github.com}.

\section{Thanks to}

\begin{itemize}
  \item {\emph{The TeXmacs developers}} for giving us such a magnificent tool.
  \item {\emph{Massimiliano Gubinelli}} for setting me straight on the help
  file mechanism for plug-ins.
\end{itemize}

\end{document}
